\documentclass[../main.tex]{subfiles}

\begin{document}

\begin{lemma}[Topologická spojitost]
    TODO
\end{lemma}
\begin{proof}
    TODO
\end{proof}

\begin{lemma}[spojitý obraz kompaktu]
Nechť $(M,d)$ a $(N,e)$ jsou metrické prostory. $X \subset M$ je neprázdná
kompaktní množina a $f: X \to N$ je spojitá funkce pak je obraz $f[X] \subset N$
je kompaktní množina.
\end{lemma}
\begin{proof}
    TODO
\end{proof}

\begin{lemma}[spojitost inverzu]
    Nechť $f: X \to N$ je spojité zobrazení z neprázdné kompaktní množiny
    $X \subset M$ v $(M,d)$ do $(N,e)$. Potom inverzní zobrazení
    $f^{-1}: f[X] \to X$ je spojité.
\end{lemma}
\begin{proof}
    TODO
\end{proof}

\begin{definition}[Homeomorfismy]
    
\end{definition}

\begin{theorem}[Heine-Borelova]
    
\end{theorem}



\end{document}
