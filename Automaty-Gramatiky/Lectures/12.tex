\documentclass[../main.tex]{subfiles}

%{definition}{Definice}
%{example}{Příklad}
%{intuition}{Intuice}
%{remark}{Poznámka}
%{consequence}{Důsledek}
%{observation}{Pozorování}

\begin{document}
%%%%%%%%%%%%%%%%%%%%%%%%%%%%%%%%%%%%%%%%%%%%%%%%%%%%%%%%%%%%%%%%%%%%%%%%%%%%%%%%
\section{Dvanáctá přednáška}

\begin{example}
    Jazyk $L = {a^{2n}|n \geq 0}$ přijíma Turingův stroj $M = ({q_0,q_1,q_F},{a},{a,B},\delta,q_0,B, {q_F})$ s 
    $\delta$ v tabulce:\\
    \textbf{TODO, 220/212-233 Example 11.2}
\end{example}
 \begin{itemize}
     \item Regulární jazyky:
     \begin{itemize}
         \item simulujeme konečný automat, pohyb hlavy vždy vpravo
         \item vidíme-li $B$, t. j. konec vstupu, přejdu do nového přijímacího stavu $q_F$.
         \item (Z přijímajících stavů nemá TM instrukci.)
     \end{itemize}
     \item Bezkontextové jazyky: nejsnáze s pomocnou páskou simulující zásobník, bude za chvíli.
 \end{itemize}
 \begin{theorem}
     Každý rekurzivně spočetný jazyk je typu 0
     \begin{proof}
         pro TM $T$ najdeme gramatiku $G, L(T) = L(G)$
         \begin{enumerate}
             \item $G = ({S,C,D,E})\cup {\underline{X}}_{x \in \Sigma \cup \Gamma} \cup {Q_i}_{q_i \in Q},\Sigma,P,S)$, P je:
             \item gramatika nejdříve vygeneruje pásku stroje a kopii slova $wB^n\underline{X}^RQ_0B^m$, kde $B^i$ představují volný prostor pro výpočet
             \item potom simuluje výpočet (stavy jsou součástí slova)
             \item v koncovém stavu smažeme pásku, necháme pouze kopii slova
         \end{enumerate}
         \textbf{TODO, 224/212-233 Proof of Theorem 11.1}
         Ještě  $L(T) = L(G)$ ?
         \begin{itemize}
             \item $w \in L(T)$
             \begin{itemize}
                 \item existuje konečný výpočet stroje $T$ (konečný prostor)
                 \item gramatika vygeneruje dostatečně velký prostor pro výpočet
                 \item simuluje výpočet a smaže dvojníky
             \end{itemize}
             \item $w \in L(G)$
             \begin{itemize}
                 \item pravidla v derivaci nemusí být v pořadí, jakým \textbf{TODO, 224}
             \end{itemize}
         \end{itemize}
     \end{proof}
 \end{theorem}
 \begin{theorem}
     Každý jazyk typu 0 je rekurzivně spočetný
     \begin{proof}
         idea: TM postupně generuje všechny derivace
         \begin{itemize}
             \item derivaci $S \Rightarrow \omega_1 \Rightarrow \dots \Rightarrow \omega_n = w$ kódujeme jako slovo $\#S\#\omega_1\#\dots\#w\#$
             \item umíme udělat TM, který:
             \begin{itemize}
                 \item přijímá slova $\#\alpha\#\beta\#$, kde $\alpha \Rightarrow \beta$
                 \item přijímá slova $\#\omega_1\#\dots\#\omega_k\#$, kde $\omega_1 \Rightarrow^* \omega_k$
                 \item postupně generuje všechny slova
             \end{itemize}
         \end{itemize}
     \end{proof}
 \end{theorem}
 \begin{definition}
     Počáteční pozice
     \begin{itemize}
         \item vstup na první pásce, ostatní zcela prázdné
         \item první hlava vlevo od vstupu, ostatní libovolně
         \item hlava v počátečním stavu
     \end{itemize}
     Jeden krok vícepáskového TM
     \begin{itemize}
         \item hlava přejde do nového stavu
         \item na každé pásce napíšeme nový symbol
         \item každá hlava se nezávisle posune vlevo, zůstane, vpravo
     \end{itemize}
 \end{definition}
 \begin{theorem}
     Každý jazyk přijímaný vícepáskovým TM je přijímaný i nějakým (jednopáskovým) Turingovým strojem TM.
     \begin{proof}
         \begin{itemize}
             \item konstruujeme TM $M$
             \item pásku si představíme, že má $2k$ stop
             \begin{itemize}
                 \item liché stopy: pozice $k$-té hlavy
                 \item sudé stopy: znak na $k$-té pásce
             \end{itemize}
             \item pro simulaci jednoho kroku navštívíme všechny hlavy
             \item ve stavu si pamatujeme
             \begin{itemize}
                 \item počet hlav vlevo
                 \item $\forall k$ symbol pod $k$-tou hlavou
             \end{itemize}
             \item pak už umíme provést jeden krok (znovu běhat)
         \end{itemize}
     \end{proof}
     Simulaci výpočtu $k$-páskového stroje o $n$ krocích lze provést v čase $O(n^2)$ (simulace jednoho kroku z
     prvních $n$ trvá $4n+2k$, hlavy nejvýš $2n$ daleko, přečíst, zapsat, posunout značky).
 \end{theorem}
\begin{definition}
    Nedeterministický TM\\
    nazýváme sedmici $M = (Q,\Sigma,\Gamma,\delta,q_0,B,F)$, kde $Q,\Sigma,\Gamma,q_0,B,F$ jsou jako u TM a 
    $\delta : (Q-F)\times \Gamma \rightarrow P(Q\times \Gamma \times {L,R})$.

    Slovo $w \in \Sigma^*$ je přijímano nedeterministickým TM $M$, pokud existuje nějaký výpočet $q_0w\vdash^*\alpha p \beta, p \in F$.
\end{definition}
\begin{theorem}
    Pro každý $M_N$ nedeterministický $TM$ existuje deterministický $TM M_D$ takový, že $L(M_N) = L(M_D)$.
    \begin{proof}
        prohledáváme do šířky možné výpočty $M_N$
        \begin{itemize}
            \item odvozeno v $k$ krocích
            \item maximálně $m$ konfigurací (kde $m = max|\delta(q,x)|$ je max. počet voleb $M_N$)
        \end{itemize}
    \end{proof} 
\end{theorem}
\end{document}